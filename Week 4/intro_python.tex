
% Default to the notebook output style

    


% Inherit from the specified cell style.




    
\documentclass[11pt]{article}

    
    
    \usepackage[T1]{fontenc}
    % Nicer default font (+ math font) than Computer Modern for most use cases
    \usepackage{mathpazo}

    % Basic figure setup, for now with no caption control since it's done
    % automatically by Pandoc (which extracts ![](path) syntax from Markdown).
    \usepackage{graphicx}
    % We will generate all images so they have a width \maxwidth. This means
    % that they will get their normal width if they fit onto the page, but
    % are scaled down if they would overflow the margins.
    \makeatletter
    \def\maxwidth{\ifdim\Gin@nat@width>\linewidth\linewidth
    \else\Gin@nat@width\fi}
    \makeatother
    \let\Oldincludegraphics\includegraphics
    % Set max figure width to be 80% of text width, for now hardcoded.
    \renewcommand{\includegraphics}[1]{\Oldincludegraphics[width=.8\maxwidth]{#1}}
    % Ensure that by default, figures have no caption (until we provide a
    % proper Figure object with a Caption API and a way to capture that
    % in the conversion process - todo).
    \usepackage{caption}
    \DeclareCaptionLabelFormat{nolabel}{}
    \captionsetup{labelformat=nolabel}

    \usepackage{adjustbox} % Used to constrain images to a maximum size 
    \usepackage{xcolor} % Allow colors to be defined
    \usepackage{enumerate} % Needed for markdown enumerations to work
    \usepackage{geometry} % Used to adjust the document margins
    \usepackage{amsmath} % Equations
    \usepackage{amssymb} % Equations
    \usepackage{textcomp} % defines textquotesingle
    % Hack from http://tex.stackexchange.com/a/47451/13684:
    \AtBeginDocument{%
        \def\PYZsq{\textquotesingle}% Upright quotes in Pygmentized code
    }
    \usepackage{upquote} % Upright quotes for verbatim code
    \usepackage{eurosym} % defines \euro
    \usepackage[mathletters]{ucs} % Extended unicode (utf-8) support
    \usepackage[utf8x]{inputenc} % Allow utf-8 characters in the tex document
    \usepackage{fancyvrb} % verbatim replacement that allows latex
    \usepackage{grffile} % extends the file name processing of package graphics 
                         % to support a larger range 
    % The hyperref package gives us a pdf with properly built
    % internal navigation ('pdf bookmarks' for the table of contents,
    % internal cross-reference links, web links for URLs, etc.)
    \usepackage{hyperref}
    \usepackage{longtable} % longtable support required by pandoc >1.10
    \usepackage{booktabs}  % table support for pandoc > 1.12.2
    \usepackage[inline]{enumitem} % IRkernel/repr support (it uses the enumerate* environment)
    \usepackage[normalem]{ulem} % ulem is needed to support strikethroughs (\sout)
                                % normalem makes italics be italics, not underlines
    

    
    
    % Colors for the hyperref package
    \definecolor{urlcolor}{rgb}{0,.145,.698}
    \definecolor{linkcolor}{rgb}{.71,0.21,0.01}
    \definecolor{citecolor}{rgb}{.12,.54,.11}

    % ANSI colors
    \definecolor{ansi-black}{HTML}{3E424D}
    \definecolor{ansi-black-intense}{HTML}{282C36}
    \definecolor{ansi-red}{HTML}{E75C58}
    \definecolor{ansi-red-intense}{HTML}{B22B31}
    \definecolor{ansi-green}{HTML}{00A250}
    \definecolor{ansi-green-intense}{HTML}{007427}
    \definecolor{ansi-yellow}{HTML}{DDB62B}
    \definecolor{ansi-yellow-intense}{HTML}{B27D12}
    \definecolor{ansi-blue}{HTML}{208FFB}
    \definecolor{ansi-blue-intense}{HTML}{0065CA}
    \definecolor{ansi-magenta}{HTML}{D160C4}
    \definecolor{ansi-magenta-intense}{HTML}{A03196}
    \definecolor{ansi-cyan}{HTML}{60C6C8}
    \definecolor{ansi-cyan-intense}{HTML}{258F8F}
    \definecolor{ansi-white}{HTML}{C5C1B4}
    \definecolor{ansi-white-intense}{HTML}{A1A6B2}

    % commands and environments needed by pandoc snippets
    % extracted from the output of `pandoc -s`
    \providecommand{\tightlist}{%
      \setlength{\itemsep}{0pt}\setlength{\parskip}{0pt}}
    \DefineVerbatimEnvironment{Highlighting}{Verbatim}{commandchars=\\\{\}}
    % Add ',fontsize=\small' for more characters per line
    \newenvironment{Shaded}{}{}
    \newcommand{\KeywordTok}[1]{\textcolor[rgb]{0.00,0.44,0.13}{\textbf{{#1}}}}
    \newcommand{\DataTypeTok}[1]{\textcolor[rgb]{0.56,0.13,0.00}{{#1}}}
    \newcommand{\DecValTok}[1]{\textcolor[rgb]{0.25,0.63,0.44}{{#1}}}
    \newcommand{\BaseNTok}[1]{\textcolor[rgb]{0.25,0.63,0.44}{{#1}}}
    \newcommand{\FloatTok}[1]{\textcolor[rgb]{0.25,0.63,0.44}{{#1}}}
    \newcommand{\CharTok}[1]{\textcolor[rgb]{0.25,0.44,0.63}{{#1}}}
    \newcommand{\StringTok}[1]{\textcolor[rgb]{0.25,0.44,0.63}{{#1}}}
    \newcommand{\CommentTok}[1]{\textcolor[rgb]{0.38,0.63,0.69}{\textit{{#1}}}}
    \newcommand{\OtherTok}[1]{\textcolor[rgb]{0.00,0.44,0.13}{{#1}}}
    \newcommand{\AlertTok}[1]{\textcolor[rgb]{1.00,0.00,0.00}{\textbf{{#1}}}}
    \newcommand{\FunctionTok}[1]{\textcolor[rgb]{0.02,0.16,0.49}{{#1}}}
    \newcommand{\RegionMarkerTok}[1]{{#1}}
    \newcommand{\ErrorTok}[1]{\textcolor[rgb]{1.00,0.00,0.00}{\textbf{{#1}}}}
    \newcommand{\NormalTok}[1]{{#1}}
    
    % Additional commands for more recent versions of Pandoc
    \newcommand{\ConstantTok}[1]{\textcolor[rgb]{0.53,0.00,0.00}{{#1}}}
    \newcommand{\SpecialCharTok}[1]{\textcolor[rgb]{0.25,0.44,0.63}{{#1}}}
    \newcommand{\VerbatimStringTok}[1]{\textcolor[rgb]{0.25,0.44,0.63}{{#1}}}
    \newcommand{\SpecialStringTok}[1]{\textcolor[rgb]{0.73,0.40,0.53}{{#1}}}
    \newcommand{\ImportTok}[1]{{#1}}
    \newcommand{\DocumentationTok}[1]{\textcolor[rgb]{0.73,0.13,0.13}{\textit{{#1}}}}
    \newcommand{\AnnotationTok}[1]{\textcolor[rgb]{0.38,0.63,0.69}{\textbf{\textit{{#1}}}}}
    \newcommand{\CommentVarTok}[1]{\textcolor[rgb]{0.38,0.63,0.69}{\textbf{\textit{{#1}}}}}
    \newcommand{\VariableTok}[1]{\textcolor[rgb]{0.10,0.09,0.49}{{#1}}}
    \newcommand{\ControlFlowTok}[1]{\textcolor[rgb]{0.00,0.44,0.13}{\textbf{{#1}}}}
    \newcommand{\OperatorTok}[1]{\textcolor[rgb]{0.40,0.40,0.40}{{#1}}}
    \newcommand{\BuiltInTok}[1]{{#1}}
    \newcommand{\ExtensionTok}[1]{{#1}}
    \newcommand{\PreprocessorTok}[1]{\textcolor[rgb]{0.74,0.48,0.00}{{#1}}}
    \newcommand{\AttributeTok}[1]{\textcolor[rgb]{0.49,0.56,0.16}{{#1}}}
    \newcommand{\InformationTok}[1]{\textcolor[rgb]{0.38,0.63,0.69}{\textbf{\textit{{#1}}}}}
    \newcommand{\WarningTok}[1]{\textcolor[rgb]{0.38,0.63,0.69}{\textbf{\textit{{#1}}}}}
    
    
    % Define a nice break command that doesn't care if a line doesn't already
    % exist.
    \def\br{\hspace*{\fill} \\* }
    % Math Jax compatability definitions
    \def\gt{>}
    \def\lt{<}
    % Document parameters
    \title{intro\_python}
    
    
    

    % Pygments definitions
    
\makeatletter
\def\PY@reset{\let\PY@it=\relax \let\PY@bf=\relax%
    \let\PY@ul=\relax \let\PY@tc=\relax%
    \let\PY@bc=\relax \let\PY@ff=\relax}
\def\PY@tok#1{\csname PY@tok@#1\endcsname}
\def\PY@toks#1+{\ifx\relax#1\empty\else%
    \PY@tok{#1}\expandafter\PY@toks\fi}
\def\PY@do#1{\PY@bc{\PY@tc{\PY@ul{%
    \PY@it{\PY@bf{\PY@ff{#1}}}}}}}
\def\PY#1#2{\PY@reset\PY@toks#1+\relax+\PY@do{#2}}

\expandafter\def\csname PY@tok@w\endcsname{\def\PY@tc##1{\textcolor[rgb]{0.73,0.73,0.73}{##1}}}
\expandafter\def\csname PY@tok@c\endcsname{\let\PY@it=\textit\def\PY@tc##1{\textcolor[rgb]{0.25,0.50,0.50}{##1}}}
\expandafter\def\csname PY@tok@cp\endcsname{\def\PY@tc##1{\textcolor[rgb]{0.74,0.48,0.00}{##1}}}
\expandafter\def\csname PY@tok@k\endcsname{\let\PY@bf=\textbf\def\PY@tc##1{\textcolor[rgb]{0.00,0.50,0.00}{##1}}}
\expandafter\def\csname PY@tok@kp\endcsname{\def\PY@tc##1{\textcolor[rgb]{0.00,0.50,0.00}{##1}}}
\expandafter\def\csname PY@tok@kt\endcsname{\def\PY@tc##1{\textcolor[rgb]{0.69,0.00,0.25}{##1}}}
\expandafter\def\csname PY@tok@o\endcsname{\def\PY@tc##1{\textcolor[rgb]{0.40,0.40,0.40}{##1}}}
\expandafter\def\csname PY@tok@ow\endcsname{\let\PY@bf=\textbf\def\PY@tc##1{\textcolor[rgb]{0.67,0.13,1.00}{##1}}}
\expandafter\def\csname PY@tok@nb\endcsname{\def\PY@tc##1{\textcolor[rgb]{0.00,0.50,0.00}{##1}}}
\expandafter\def\csname PY@tok@nf\endcsname{\def\PY@tc##1{\textcolor[rgb]{0.00,0.00,1.00}{##1}}}
\expandafter\def\csname PY@tok@nc\endcsname{\let\PY@bf=\textbf\def\PY@tc##1{\textcolor[rgb]{0.00,0.00,1.00}{##1}}}
\expandafter\def\csname PY@tok@nn\endcsname{\let\PY@bf=\textbf\def\PY@tc##1{\textcolor[rgb]{0.00,0.00,1.00}{##1}}}
\expandafter\def\csname PY@tok@ne\endcsname{\let\PY@bf=\textbf\def\PY@tc##1{\textcolor[rgb]{0.82,0.25,0.23}{##1}}}
\expandafter\def\csname PY@tok@nv\endcsname{\def\PY@tc##1{\textcolor[rgb]{0.10,0.09,0.49}{##1}}}
\expandafter\def\csname PY@tok@no\endcsname{\def\PY@tc##1{\textcolor[rgb]{0.53,0.00,0.00}{##1}}}
\expandafter\def\csname PY@tok@nl\endcsname{\def\PY@tc##1{\textcolor[rgb]{0.63,0.63,0.00}{##1}}}
\expandafter\def\csname PY@tok@ni\endcsname{\let\PY@bf=\textbf\def\PY@tc##1{\textcolor[rgb]{0.60,0.60,0.60}{##1}}}
\expandafter\def\csname PY@tok@na\endcsname{\def\PY@tc##1{\textcolor[rgb]{0.49,0.56,0.16}{##1}}}
\expandafter\def\csname PY@tok@nt\endcsname{\let\PY@bf=\textbf\def\PY@tc##1{\textcolor[rgb]{0.00,0.50,0.00}{##1}}}
\expandafter\def\csname PY@tok@nd\endcsname{\def\PY@tc##1{\textcolor[rgb]{0.67,0.13,1.00}{##1}}}
\expandafter\def\csname PY@tok@s\endcsname{\def\PY@tc##1{\textcolor[rgb]{0.73,0.13,0.13}{##1}}}
\expandafter\def\csname PY@tok@sd\endcsname{\let\PY@it=\textit\def\PY@tc##1{\textcolor[rgb]{0.73,0.13,0.13}{##1}}}
\expandafter\def\csname PY@tok@si\endcsname{\let\PY@bf=\textbf\def\PY@tc##1{\textcolor[rgb]{0.73,0.40,0.53}{##1}}}
\expandafter\def\csname PY@tok@se\endcsname{\let\PY@bf=\textbf\def\PY@tc##1{\textcolor[rgb]{0.73,0.40,0.13}{##1}}}
\expandafter\def\csname PY@tok@sr\endcsname{\def\PY@tc##1{\textcolor[rgb]{0.73,0.40,0.53}{##1}}}
\expandafter\def\csname PY@tok@ss\endcsname{\def\PY@tc##1{\textcolor[rgb]{0.10,0.09,0.49}{##1}}}
\expandafter\def\csname PY@tok@sx\endcsname{\def\PY@tc##1{\textcolor[rgb]{0.00,0.50,0.00}{##1}}}
\expandafter\def\csname PY@tok@m\endcsname{\def\PY@tc##1{\textcolor[rgb]{0.40,0.40,0.40}{##1}}}
\expandafter\def\csname PY@tok@gh\endcsname{\let\PY@bf=\textbf\def\PY@tc##1{\textcolor[rgb]{0.00,0.00,0.50}{##1}}}
\expandafter\def\csname PY@tok@gu\endcsname{\let\PY@bf=\textbf\def\PY@tc##1{\textcolor[rgb]{0.50,0.00,0.50}{##1}}}
\expandafter\def\csname PY@tok@gd\endcsname{\def\PY@tc##1{\textcolor[rgb]{0.63,0.00,0.00}{##1}}}
\expandafter\def\csname PY@tok@gi\endcsname{\def\PY@tc##1{\textcolor[rgb]{0.00,0.63,0.00}{##1}}}
\expandafter\def\csname PY@tok@gr\endcsname{\def\PY@tc##1{\textcolor[rgb]{1.00,0.00,0.00}{##1}}}
\expandafter\def\csname PY@tok@ge\endcsname{\let\PY@it=\textit}
\expandafter\def\csname PY@tok@gs\endcsname{\let\PY@bf=\textbf}
\expandafter\def\csname PY@tok@gp\endcsname{\let\PY@bf=\textbf\def\PY@tc##1{\textcolor[rgb]{0.00,0.00,0.50}{##1}}}
\expandafter\def\csname PY@tok@go\endcsname{\def\PY@tc##1{\textcolor[rgb]{0.53,0.53,0.53}{##1}}}
\expandafter\def\csname PY@tok@gt\endcsname{\def\PY@tc##1{\textcolor[rgb]{0.00,0.27,0.87}{##1}}}
\expandafter\def\csname PY@tok@err\endcsname{\def\PY@bc##1{\setlength{\fboxsep}{0pt}\fcolorbox[rgb]{1.00,0.00,0.00}{1,1,1}{\strut ##1}}}
\expandafter\def\csname PY@tok@kc\endcsname{\let\PY@bf=\textbf\def\PY@tc##1{\textcolor[rgb]{0.00,0.50,0.00}{##1}}}
\expandafter\def\csname PY@tok@kd\endcsname{\let\PY@bf=\textbf\def\PY@tc##1{\textcolor[rgb]{0.00,0.50,0.00}{##1}}}
\expandafter\def\csname PY@tok@kn\endcsname{\let\PY@bf=\textbf\def\PY@tc##1{\textcolor[rgb]{0.00,0.50,0.00}{##1}}}
\expandafter\def\csname PY@tok@kr\endcsname{\let\PY@bf=\textbf\def\PY@tc##1{\textcolor[rgb]{0.00,0.50,0.00}{##1}}}
\expandafter\def\csname PY@tok@bp\endcsname{\def\PY@tc##1{\textcolor[rgb]{0.00,0.50,0.00}{##1}}}
\expandafter\def\csname PY@tok@fm\endcsname{\def\PY@tc##1{\textcolor[rgb]{0.00,0.00,1.00}{##1}}}
\expandafter\def\csname PY@tok@vc\endcsname{\def\PY@tc##1{\textcolor[rgb]{0.10,0.09,0.49}{##1}}}
\expandafter\def\csname PY@tok@vg\endcsname{\def\PY@tc##1{\textcolor[rgb]{0.10,0.09,0.49}{##1}}}
\expandafter\def\csname PY@tok@vi\endcsname{\def\PY@tc##1{\textcolor[rgb]{0.10,0.09,0.49}{##1}}}
\expandafter\def\csname PY@tok@vm\endcsname{\def\PY@tc##1{\textcolor[rgb]{0.10,0.09,0.49}{##1}}}
\expandafter\def\csname PY@tok@sa\endcsname{\def\PY@tc##1{\textcolor[rgb]{0.73,0.13,0.13}{##1}}}
\expandafter\def\csname PY@tok@sb\endcsname{\def\PY@tc##1{\textcolor[rgb]{0.73,0.13,0.13}{##1}}}
\expandafter\def\csname PY@tok@sc\endcsname{\def\PY@tc##1{\textcolor[rgb]{0.73,0.13,0.13}{##1}}}
\expandafter\def\csname PY@tok@dl\endcsname{\def\PY@tc##1{\textcolor[rgb]{0.73,0.13,0.13}{##1}}}
\expandafter\def\csname PY@tok@s2\endcsname{\def\PY@tc##1{\textcolor[rgb]{0.73,0.13,0.13}{##1}}}
\expandafter\def\csname PY@tok@sh\endcsname{\def\PY@tc##1{\textcolor[rgb]{0.73,0.13,0.13}{##1}}}
\expandafter\def\csname PY@tok@s1\endcsname{\def\PY@tc##1{\textcolor[rgb]{0.73,0.13,0.13}{##1}}}
\expandafter\def\csname PY@tok@mb\endcsname{\def\PY@tc##1{\textcolor[rgb]{0.40,0.40,0.40}{##1}}}
\expandafter\def\csname PY@tok@mf\endcsname{\def\PY@tc##1{\textcolor[rgb]{0.40,0.40,0.40}{##1}}}
\expandafter\def\csname PY@tok@mh\endcsname{\def\PY@tc##1{\textcolor[rgb]{0.40,0.40,0.40}{##1}}}
\expandafter\def\csname PY@tok@mi\endcsname{\def\PY@tc##1{\textcolor[rgb]{0.40,0.40,0.40}{##1}}}
\expandafter\def\csname PY@tok@il\endcsname{\def\PY@tc##1{\textcolor[rgb]{0.40,0.40,0.40}{##1}}}
\expandafter\def\csname PY@tok@mo\endcsname{\def\PY@tc##1{\textcolor[rgb]{0.40,0.40,0.40}{##1}}}
\expandafter\def\csname PY@tok@ch\endcsname{\let\PY@it=\textit\def\PY@tc##1{\textcolor[rgb]{0.25,0.50,0.50}{##1}}}
\expandafter\def\csname PY@tok@cm\endcsname{\let\PY@it=\textit\def\PY@tc##1{\textcolor[rgb]{0.25,0.50,0.50}{##1}}}
\expandafter\def\csname PY@tok@cpf\endcsname{\let\PY@it=\textit\def\PY@tc##1{\textcolor[rgb]{0.25,0.50,0.50}{##1}}}
\expandafter\def\csname PY@tok@c1\endcsname{\let\PY@it=\textit\def\PY@tc##1{\textcolor[rgb]{0.25,0.50,0.50}{##1}}}
\expandafter\def\csname PY@tok@cs\endcsname{\let\PY@it=\textit\def\PY@tc##1{\textcolor[rgb]{0.25,0.50,0.50}{##1}}}

\def\PYZbs{\char`\\}
\def\PYZus{\char`\_}
\def\PYZob{\char`\{}
\def\PYZcb{\char`\}}
\def\PYZca{\char`\^}
\def\PYZam{\char`\&}
\def\PYZlt{\char`\<}
\def\PYZgt{\char`\>}
\def\PYZsh{\char`\#}
\def\PYZpc{\char`\%}
\def\PYZdl{\char`\$}
\def\PYZhy{\char`\-}
\def\PYZsq{\char`\'}
\def\PYZdq{\char`\"}
\def\PYZti{\char`\~}
% for compatibility with earlier versions
\def\PYZat{@}
\def\PYZlb{[}
\def\PYZrb{]}
\makeatother


    % Exact colors from NB
    \definecolor{incolor}{rgb}{0.0, 0.0, 0.5}
    \definecolor{outcolor}{rgb}{0.545, 0.0, 0.0}



    
    % Prevent overflowing lines due to hard-to-break entities
    \sloppy 
    % Setup hyperref package
    \hypersetup{
      breaklinks=true,  % so long urls are correctly broken across lines
      colorlinks=true,
      urlcolor=urlcolor,
      linkcolor=linkcolor,
      citecolor=citecolor,
      }
    % Slightly bigger margins than the latex defaults
    
    \geometry{verbose,tmargin=1in,bmargin=1in,lmargin=1in,rmargin=1in}
    
    

    \begin{document}
    
    
    \maketitle
    
    

    
    \section{INTRODUCTION TO PYTHON}\label{introduction-to-python}

    \subsection{Description}\label{description}

Python is a widely used high-level, general-purpose, interpreted,
dynamic programming language. Its design philosophy emphasizes code
readability, and its syntax allows programmers to express concepts in
fewer lines of code than would be possible in languages such as C++ or
Java. The language provides constructs intended to enable clear programs
on both a small and large scale. (Source:
https://anaconda.org/anaconda/python)

    \subsection{General advantages:}\label{general-advantages}

\begin{itemize}
\tightlist
\item
  Easy to use.
\item
  High-level programming.
\item
  Requires less lines of coding compared to other languages.
\item
  Large online community for support.
\item
  Capability of porting scripts to other platforms without Python
  interpreters.
\end{itemize}

    \subsection{Specific advantages:}\label{specific-advantages}

\begin{itemize}
\tightlist
\item
  Preinstalled in UNIX systems.
\item
  Capability to import more than 1000 pre-designed modules for
  penetration testing such as ``hashlib''.
\item
  Capability to quickly build scripts to deliver exploits, manipulate
  well-known network protocols and create custom packets.
\item
  Interactive shell lets programmer figure out how a specific function
  is working.
\item
  Implementation of image recognition for biometrics.
\end{itemize}

    \subsection{Some statistics}\label{some-statistics}

\begin{itemize}
\item
  Python is the 2nd most demanded programming language in 2018
  (https://www.codingdojo.com/blog/7-most-in-demand-programming-languages-of-2018/)
\item
  It is the 2nd best paid programming skill in the market
  (https://medium.freecodecamp.org/best-programming-languages-to-learn-in-2018-ultimate-guide-bfc93e615b35)
\item
  It is the best ranked programmin language according to IEEE
  (https://spectrum.ieee.org/at-work/innovation/the-2018-top-programming-languages)
\end{itemize}

    \section{How to install Python?}\label{how-to-install-python}

    \subsection{Option 1:}\label{option-1}

\begin{itemize}
\tightlist
\item
  Install Python from https://www.python.org
\item
  Install a Python Integrated Development Environment (IDE) such as
  Pycharm (available at https://www.jetbrains.com/pycharm/)
\item
  Install Jupytor Notebook, available at http://jupyter.org/
\end{itemize}

    \subsection{\texorpdfstring{Option 2 (\textbf{Better
Solution}):}{Option 2 (Better Solution):}}\label{option-2-better-solution}

All can be easily installed in a bundle called Anaconda
https://www.anaconda.com/download/. Anaconda contains an IDE called
Spyder and it even contains R!

    \subsection{Python in Linux}\label{python-in-linux}

Also, keep in mind that Ubuntu already contains Python 2. To update to
Python 3, use the command:

\begin{itemize}
\tightlist
\item
  sudo apt-get python3
\end{itemize}

    \section{Where to learn Python?}\label{where-to-learn-python}

    \subsection{Books:}\label{books}

\begin{itemize}
\tightlist
\item
  Mark Lutz, David Ascher (1999). "Learning Python".
\end{itemize}

    \subsection{Forums:}\label{forums}

\begin{itemize}
\tightlist
\item
  StackOverflow https://stackoverflow.com/questions/tagged/python The
  online community is always a great source of help.
\end{itemize}

    \subsection{Online}\label{online}

\begin{itemize}
\tightlist
\item
  https://docs.python.org/3/tutorial/
\item
  Edx: They have a free course called "Python in Research"
  (https://www.edx.org/course/using-python-research-harvardx-ph526x-0)
  by Harvard University where you can get a certificate.
\item
  DataCamp has developed the site learnpython.org
  (https://www.learnpython.org/en/). It contains an very complete online
  tutorial which even facilitates an online python IDE where you can run
  code and test it.
\item
  Courses on Coursera and Datacamp.
\end{itemize}

    \subsection{Practice!}\label{practice}

\begin{itemize}
\tightlist
\item
  DataCamp also has a mobile app with quick and interactive courses on
  Python.
\end{itemize}

    \section{Python Data Structures}\label{python-data-structures}

    \subsection{Numbers}\label{numbers}

Python's most basic function is as a calculator. It handles multiple
types of numbers such as integers, floating point or boolean operators.

    \begin{Verbatim}[commandchars=\\\{\}]
{\color{incolor}In [{\color{incolor}1}]:} \PY{l+m+mi}{8}\PY{o}{+}\PY{l+m+mi}{9}
\end{Verbatim}


\begin{Verbatim}[commandchars=\\\{\}]
{\color{outcolor}Out[{\color{outcolor}1}]:} 17
\end{Verbatim}
            
    \begin{Verbatim}[commandchars=\\\{\}]
{\color{incolor}In [{\color{incolor}2}]:} \PY{l+m+mi}{5}\PY{o}{\PYZhy{}}\PY{l+m+mf}{10.9}\PY{o}{/}\PY{l+m+mi}{3}
\end{Verbatim}


\begin{Verbatim}[commandchars=\\\{\}]
{\color{outcolor}Out[{\color{outcolor}2}]:} 1.3666666666666667
\end{Verbatim}
            
    \begin{Verbatim}[commandchars=\\\{\}]
{\color{incolor}In [{\color{incolor}3}]:} \PY{k+kc}{True} \PY{o+ow}{or} \PY{k+kc}{False}
\end{Verbatim}


\begin{Verbatim}[commandchars=\\\{\}]
{\color{outcolor}Out[{\color{outcolor}3}]:} True
\end{Verbatim}
            
    Numbers can be assigned to variables using "=".

    \begin{Verbatim}[commandchars=\\\{\}]
{\color{incolor}In [{\color{incolor}4}]:} \PY{n}{X} \PY{o}{=} \PY{l+m+mi}{2}
        \PY{n}{Y} \PY{o}{=} \PY{n}{X}\PY{o}{*}\PY{o}{*}\PY{l+m+mi}{2} \PY{o}{+} \PY{l+m+mi}{10}
        \PY{n+nb}{print}\PY{p}{(}\PY{l+s+s1}{\PYZsq{}}\PY{l+s+s1}{Result is}\PY{l+s+s1}{\PYZsq{}}\PY{p}{,} \PY{n}{Y}\PY{p}{)}
\end{Verbatim}


    \begin{Verbatim}[commandchars=\\\{\}]
Result is 14

    \end{Verbatim}

    By default, when defining a variable and storing a number, the
variable's \textbf{type} will be integer (if the number has no
decimals). Otherwise, the variable will contain a float.

    \begin{Verbatim}[commandchars=\\\{\}]
{\color{incolor}In [{\color{incolor}5}]:} \PY{n}{x} \PY{o}{=} \PY{l+m+mi}{1}
        \PY{n+nb}{type}\PY{p}{(}\PY{n}{x}\PY{p}{)}
\end{Verbatim}


\begin{Verbatim}[commandchars=\\\{\}]
{\color{outcolor}Out[{\color{outcolor}5}]:} int
\end{Verbatim}
            
    \begin{Verbatim}[commandchars=\\\{\}]
{\color{incolor}In [{\color{incolor}6}]:} \PY{n}{y} \PY{o}{=} \PY{l+m+mf}{2.1}
        \PY{n+nb}{type}\PY{p}{(}\PY{n}{y}\PY{p}{)}
\end{Verbatim}


\begin{Verbatim}[commandchars=\\\{\}]
{\color{outcolor}Out[{\color{outcolor}6}]:} float
\end{Verbatim}
            
    A variable containing an integer can be converted into a float and
viceversa.

    \begin{Verbatim}[commandchars=\\\{\}]
{\color{incolor}In [{\color{incolor}7}]:} \PY{n}{x}\PY{o}{=}\PY{n+nb}{float}\PY{p}{(}\PY{n}{x}\PY{p}{)}
        \PY{n+nb}{print}\PY{p}{(}\PY{n}{x}\PY{p}{,} \PY{n+nb}{type}\PY{p}{(}\PY{n}{x}\PY{p}{)}\PY{p}{)}
\end{Verbatim}


    \begin{Verbatim}[commandchars=\\\{\}]
1.0 <class 'float'>

    \end{Verbatim}

    \begin{Verbatim}[commandchars=\\\{\}]
{\color{incolor}In [{\color{incolor}8}]:} \PY{n}{y}\PY{o}{=}\PY{n+nb}{int}\PY{p}{(}\PY{n}{y}\PY{p}{)}
        \PY{n+nb}{print}\PY{p}{(}\PY{n}{y}\PY{p}{,} \PY{n+nb}{type}\PY{p}{(}\PY{n}{y}\PY{p}{)}\PY{p}{)}
\end{Verbatim}


    \begin{Verbatim}[commandchars=\\\{\}]
2 <class 'int'>

    \end{Verbatim}

    \subsection{Strings}\label{strings}

Variables can also store strings (words). Strings are treated as a
"list" of characters.

    \begin{Verbatim}[commandchars=\\\{\}]
{\color{incolor}In [{\color{incolor}9}]:} \PY{n}{Uni} \PY{o}{=} \PY{l+s+s2}{\PYZdq{}}\PY{l+s+s2}{Robert Gordon University}\PY{l+s+s2}{\PYZdq{}}
        \PY{n+nb}{print}\PY{p}{(}\PY{n}{Uni}\PY{p}{)}
\end{Verbatim}


    \begin{Verbatim}[commandchars=\\\{\}]
Robert Gordon University

    \end{Verbatim}

    Indexing: Python considers the first value to be in position \textbf{0}
and the last one to be \textbf{-1}.

    \begin{Verbatim}[commandchars=\\\{\}]
{\color{incolor}In [{\color{incolor}10}]:} \PY{n+nb}{len}\PY{p}{(}\PY{n}{Uni}\PY{p}{)}
         \PY{n+nb}{print}\PY{p}{(}\PY{l+s+s1}{\PYZsq{}}\PY{l+s+s1}{Length of the string Uni is: }\PY{l+s+s1}{\PYZsq{}}\PY{p}{,}\PY{n+nb}{len}\PY{p}{(}\PY{n}{Uni}\PY{p}{)}\PY{p}{)}
         \PY{n+nb}{print}\PY{p}{(}\PY{l+s+s1}{\PYZsq{}}\PY{l+s+s1}{First character of Uni is: }\PY{l+s+s1}{\PYZsq{}}\PY{p}{,} \PY{n}{Uni}\PY{p}{[}\PY{l+m+mi}{0}\PY{p}{]}\PY{p}{)}
         \PY{n+nb}{print}\PY{p}{(}\PY{l+s+s1}{\PYZsq{}}\PY{l+s+s1}{Last character of Uni is: }\PY{l+s+s1}{\PYZsq{}}\PY{p}{,}\PY{n}{Uni}\PY{p}{[}\PY{o}{\PYZhy{}}\PY{l+m+mi}{1}\PY{p}{]}\PY{p}{)}
\end{Verbatim}


    \begin{Verbatim}[commandchars=\\\{\}]
Length of the string Uni is:  24
First character of Uni is:  R
Last character of Uni is:  y

    \end{Verbatim}

    When requesting a range of characters, python "ignores" the last one.

    \begin{Verbatim}[commandchars=\\\{\}]
{\color{incolor}In [{\color{incolor}11}]:} \PY{n}{Uni}\PY{p}{[}\PY{l+m+mi}{0}\PY{p}{:}\PY{l+m+mi}{2}\PY{p}{]} \PY{c+c1}{\PYZsh{} characters from position 0 (included) to 2 (excluded)}
\end{Verbatim}


\begin{Verbatim}[commandchars=\\\{\}]
{\color{outcolor}Out[{\color{outcolor}11}]:} 'Ro'
\end{Verbatim}
            
    \begin{Verbatim}[commandchars=\\\{\}]
{\color{incolor}In [{\color{incolor}12}]:} \PY{n}{Uni}\PY{p}{[}\PY{l+m+mi}{2}\PY{p}{:}\PY{l+m+mi}{5}\PY{p}{]}  \PY{c+c1}{\PYZsh{} characters from position 2 (included) to 5 (excluded)}
\end{Verbatim}


\begin{Verbatim}[commandchars=\\\{\}]
{\color{outcolor}Out[{\color{outcolor}12}]:} 'ber'
\end{Verbatim}
            
    \begin{Verbatim}[commandchars=\\\{\}]
{\color{incolor}In [{\color{incolor}13}]:} \PY{n}{Uni}\PY{p}{[}\PY{p}{:}\PY{l+m+mi}{5}\PY{p}{]}
\end{Verbatim}


\begin{Verbatim}[commandchars=\\\{\}]
{\color{outcolor}Out[{\color{outcolor}13}]:} 'Rober'
\end{Verbatim}
            
    \begin{Verbatim}[commandchars=\\\{\}]
{\color{incolor}In [{\color{incolor}14}]:} \PY{n}{Uni}\PY{p}{[}\PY{l+m+mi}{2}\PY{p}{:}\PY{p}{]}
\end{Verbatim}


\begin{Verbatim}[commandchars=\\\{\}]
{\color{outcolor}Out[{\color{outcolor}14}]:} 'bert Gordon University'
\end{Verbatim}
            
    \begin{Verbatim}[commandchars=\\\{\}]
{\color{incolor}In [{\color{incolor}15}]:} \PY{n}{Uni}\PY{p}{[}\PY{o}{\PYZhy{}}\PY{l+m+mi}{2}\PY{p}{:}\PY{p}{]}\PY{c+c1}{\PYZsh{} last two characters }
\end{Verbatim}


\begin{Verbatim}[commandchars=\\\{\}]
{\color{outcolor}Out[{\color{outcolor}15}]:} 'ty'
\end{Verbatim}
            
    Just as with variables containing numbers, we can do operations with
variables containing strings. For instance, we can add a string to
another.

    \begin{Verbatim}[commandchars=\\\{\}]
{\color{incolor}In [{\color{incolor}16}]:} \PY{n}{Uni} \PY{o}{+} \PY{l+s+s1}{\PYZsq{}}\PY{l+s+s1}{ Data Science}\PY{l+s+s1}{\PYZsq{}}
\end{Verbatim}


\begin{Verbatim}[commandchars=\\\{\}]
{\color{outcolor}Out[{\color{outcolor}16}]:} 'Robert Gordon University Data Science'
\end{Verbatim}
            
    \subsection{Lists}\label{lists}

Lists are \textbf{mutable} structures which contain items of equal or
different types. Lists can even contain other lists within.

    \begin{Verbatim}[commandchars=\\\{\}]
{\color{incolor}In [{\color{incolor}17}]:} \PY{n}{squares} \PY{o}{=} \PY{p}{[}\PY{l+m+mi}{1}\PY{p}{,} \PY{l+m+mi}{4}\PY{p}{,} \PY{l+m+mi}{9}\PY{p}{,} \PY{l+m+mi}{16}\PY{p}{,} \PY{l+m+mi}{25}\PY{p}{]}
         \PY{n+nb}{print}\PY{p}{(}\PY{n}{squares}\PY{p}{,} \PY{n+nb}{type}\PY{p}{(}\PY{n}{squares}\PY{p}{)}\PY{p}{)}
\end{Verbatim}


    \begin{Verbatim}[commandchars=\\\{\}]
[1, 4, 9, 16, 25] <class 'list'>

    \end{Verbatim}

    \begin{Verbatim}[commandchars=\\\{\}]
{\color{incolor}In [{\color{incolor}18}]:} \PY{n+nb}{len}\PY{p}{(}\PY{n}{squares}\PY{p}{)}
\end{Verbatim}


\begin{Verbatim}[commandchars=\\\{\}]
{\color{outcolor}Out[{\color{outcolor}18}]:} 5
\end{Verbatim}
            
    \begin{Verbatim}[commandchars=\\\{\}]
{\color{incolor}In [{\color{incolor}19}]:} \PY{n}{things} \PY{o}{=} \PY{p}{[}\PY{l+m+mi}{1}\PY{p}{,} \PY{l+s+s1}{\PYZsq{}}\PY{l+s+s1}{John}\PY{l+s+s1}{\PYZsq{}}\PY{p}{,} \PY{l+m+mf}{9.86}\PY{p}{,} \PY{k+kc}{False}\PY{p}{,} \PY{n}{squares}\PY{p}{]}
         \PY{n}{things}
\end{Verbatim}


\begin{Verbatim}[commandchars=\\\{\}]
{\color{outcolor}Out[{\color{outcolor}19}]:} [1, 'John', 9.86, False, [1, 4, 9, 16, 25]]
\end{Verbatim}
            
    Similar to strings, we can request certain parts of the content of a
list.

    \begin{Verbatim}[commandchars=\\\{\}]
{\color{incolor}In [{\color{incolor}20}]:} \PY{n}{squares}\PY{p}{[}\PY{l+m+mi}{0}\PY{p}{]} \PY{c+c1}{\PYZsh{} returns first item}
\end{Verbatim}


\begin{Verbatim}[commandchars=\\\{\}]
{\color{outcolor}Out[{\color{outcolor}20}]:} 1
\end{Verbatim}
            
    \begin{Verbatim}[commandchars=\\\{\}]
{\color{incolor}In [{\color{incolor}21}]:} \PY{n}{squares}\PY{p}{[}\PY{p}{:}\PY{l+m+mi}{2}\PY{p}{]} \PY{c+c1}{\PYZsh{} returns items at position 0,1 (2 is exluded)}
\end{Verbatim}


\begin{Verbatim}[commandchars=\\\{\}]
{\color{outcolor}Out[{\color{outcolor}21}]:} [1, 4]
\end{Verbatim}
            
    Operations for lists.

    \begin{Verbatim}[commandchars=\\\{\}]
{\color{incolor}In [{\color{incolor}22}]:} \PY{c+c1}{\PYZsh{} add elements to the list}
         \PY{n}{squares} \PY{o}{=} \PY{n}{squares} \PY{o}{+} \PY{p}{[}\PY{l+m+mi}{36}\PY{p}{,} \PY{l+m+mi}{49}\PY{p}{]}
         \PY{n}{squares} 
\end{Verbatim}


\begin{Verbatim}[commandchars=\\\{\}]
{\color{outcolor}Out[{\color{outcolor}22}]:} [1, 4, 9, 16, 25, 36, 49]
\end{Verbatim}
            
    \begin{Verbatim}[commandchars=\\\{\}]
{\color{incolor}In [{\color{incolor}23}]:} \PY{c+c1}{\PYZsh{} Appending data to a list}
         \PY{n}{things}\PY{o}{.}\PY{n}{append}\PY{p}{(}\PY{p}{[}\PY{l+s+s1}{\PYZsq{}}\PY{l+s+s1}{a}\PY{l+s+s1}{\PYZsq{}}\PY{p}{,}\PY{l+s+s1}{\PYZsq{}}\PY{l+s+s1}{b}\PY{l+s+s1}{\PYZsq{}}\PY{p}{]}\PY{p}{)}
         \PY{n}{things}
\end{Verbatim}


\begin{Verbatim}[commandchars=\\\{\}]
{\color{outcolor}Out[{\color{outcolor}23}]:} [1, 'John', 9.86, False, [1, 4, 9, 16, 25], ['a', 'b']]
\end{Verbatim}
            
    \begin{Verbatim}[commandchars=\\\{\}]
{\color{incolor}In [{\color{incolor}24}]:} \PY{c+c1}{\PYZsh{} Compare two lists}
         \PY{p}{[}\PY{l+m+mi}{1}\PY{p}{,}\PY{l+m+mi}{2}\PY{p}{,}\PY{l+m+mi}{3}\PY{p}{]} \PY{o}{==} \PY{p}{[}\PY{l+m+mi}{1}\PY{p}{,}\PY{l+m+mi}{3}\PY{p}{,}\PY{l+m+mi}{3}\PY{p}{]}
\end{Verbatim}


\begin{Verbatim}[commandchars=\\\{\}]
{\color{outcolor}Out[{\color{outcolor}24}]:} False
\end{Verbatim}
            
    \begin{Verbatim}[commandchars=\\\{\}]
{\color{incolor}In [{\color{incolor}25}]:} \PY{c+c1}{\PYZsh{} Check if an element is in the list}
         \PY{l+m+mi}{4} \PY{o+ow}{in} \PY{n}{squares}
\end{Verbatim}


\begin{Verbatim}[commandchars=\\\{\}]
{\color{outcolor}Out[{\color{outcolor}25}]:} True
\end{Verbatim}
            
    \begin{Verbatim}[commandchars=\\\{\}]
{\color{incolor}In [{\color{incolor}26}]:} \PY{c+c1}{\PYZsh{} replace elements from the list}
         \PY{n}{squares}\PY{p}{[}\PY{l+m+mi}{2}\PY{p}{]}\PY{o}{=}\PY{l+m+mi}{12}
         \PY{n+nb}{print}\PY{p}{(}\PY{n}{squares}\PY{p}{)}
\end{Verbatim}


    \begin{Verbatim}[commandchars=\\\{\}]
[1, 4, 12, 16, 25, 36, 49]

    \end{Verbatim}

    \begin{Verbatim}[commandchars=\\\{\}]
{\color{incolor}In [{\color{incolor}27}]:} \PY{c+c1}{\PYZsh{} delete elements from the list }
         \PY{k}{del} \PY{n}{squares}\PY{p}{[}\PY{l+m+mi}{2}\PY{p}{]}
         \PY{n}{squares}
\end{Verbatim}


\begin{Verbatim}[commandchars=\\\{\}]
{\color{outcolor}Out[{\color{outcolor}27}]:} [1, 4, 16, 25, 36, 49]
\end{Verbatim}
            
    \subsection{Tuples}\label{tuples}

Similar to a list, but \textbf{immutable}.

    \begin{Verbatim}[commandchars=\\\{\}]
{\color{incolor}In [{\color{incolor}28}]:} \PY{n}{tup} \PY{o}{=} \PY{p}{(}\PY{l+m+mi}{1}\PY{p}{,}\PY{l+m+mi}{2}\PY{p}{,}\PY{l+m+mi}{3}\PY{p}{,}\PY{l+s+s1}{\PYZsq{}}\PY{l+s+s1}{Data Science}\PY{l+s+s1}{\PYZsq{}}\PY{p}{)}
         \PY{n}{tup}
\end{Verbatim}


\begin{Verbatim}[commandchars=\\\{\}]
{\color{outcolor}Out[{\color{outcolor}28}]:} (1, 2, 3, 'Data Science')
\end{Verbatim}
            
    \begin{Verbatim}[commandchars=\\\{\}]
{\color{incolor}In [{\color{incolor}29}]:} \PY{n}{tup}\PY{p}{[}\PY{l+m+mi}{1}\PY{p}{]}
\end{Verbatim}


\begin{Verbatim}[commandchars=\\\{\}]
{\color{outcolor}Out[{\color{outcolor}29}]:} 2
\end{Verbatim}
            
    \begin{Verbatim}[commandchars=\\\{\}]
{\color{incolor}In [{\color{incolor}30}]:} \PY{n+nb}{len}\PY{p}{(}\PY{n}{tup}\PY{p}{)}
\end{Verbatim}


\begin{Verbatim}[commandchars=\\\{\}]
{\color{outcolor}Out[{\color{outcolor}30}]:} 4
\end{Verbatim}
            
    \begin{Verbatim}[commandchars=\\\{\}]
{\color{incolor}In [{\color{incolor}31}]:} \PY{n}{tup}\PY{p}{[}\PY{l+m+mi}{2}\PY{p}{]}\PY{o}{=}\PY{l+m+mi}{7}
\end{Verbatim}


    \begin{Verbatim}[commandchars=\\\{\}]

        ---------------------------------------------------------------------------

        TypeError                                 Traceback (most recent call last)

        <ipython-input-31-1f03b11b5d1e> in <module>()
    ----> 1 tup[2]=7
    

        TypeError: 'tuple' object does not support item assignment

    \end{Verbatim}

    A tuple cannot be modified, but the values can be unpacked into
variables.

    \begin{Verbatim}[commandchars=\\\{\}]
{\color{incolor}In [{\color{incolor} }]:} \PY{n}{num1}\PY{p}{,}\PY{n}{num2}\PY{p}{,}\PY{n}{num3}\PY{p}{,}\PY{n}{mystring} \PY{o}{=} \PY{n}{tup}
        \PY{n+nb}{print}\PY{p}{(}\PY{n}{num1}\PY{p}{,}\PY{n}{num2}\PY{p}{,}\PY{n}{num3}\PY{p}{,}\PY{n}{mystring}\PY{p}{)}
\end{Verbatim}


    Example of using lists and tuples.

    \begin{Verbatim}[commandchars=\\\{\}]
{\color{incolor}In [{\color{incolor}32}]:} \PY{n}{list\PYZus{}physics} \PY{o}{=} \PY{p}{[}\PY{p}{(}\PY{l+s+s2}{\PYZdq{}}\PY{l+s+s2}{gravity\PYZus{}earth}\PY{l+s+s2}{\PYZdq{}}\PY{p}{,}\PY{l+m+mf}{9.81}\PY{p}{)}\PY{p}{,}\PY{p}{(}\PY{l+s+s2}{\PYZdq{}}\PY{l+s+s2}{e}\PY{l+s+s2}{\PYZdq{}}\PY{p}{,}\PY{l+m+mf}{2.3}\PY{p}{)}\PY{p}{,}\PY{p}{(}\PY{l+s+s2}{\PYZdq{}}\PY{l+s+s2}{pi}\PY{l+s+s2}{\PYZdq{}}\PY{p}{,}\PY{l+m+mf}{3.1416}\PY{p}{)}\PY{p}{]}
         \PY{n}{list\PYZus{}physics}
\end{Verbatim}


\begin{Verbatim}[commandchars=\\\{\}]
{\color{outcolor}Out[{\color{outcolor}32}]:} [('gravity\_earth', 9.81), ('e', 2.3), ('pi', 3.1416)]
\end{Verbatim}
            
    \begin{Verbatim}[commandchars=\\\{\}]
{\color{incolor}In [{\color{incolor}33}]:} \PY{n}{list\PYZus{}physics}\PY{p}{[}\PY{l+m+mi}{0}\PY{p}{]}\PY{o}{=}\PY{p}{(}\PY{l+s+s2}{\PYZdq{}}\PY{l+s+s2}{gravity\PYZus{}moon}\PY{l+s+s2}{\PYZdq{}}\PY{p}{,}\PY{l+m+mf}{9.81}\PY{o}{/}\PY{l+m+mi}{6}\PY{p}{)}
         \PY{n}{list\PYZus{}physics}
\end{Verbatim}


\begin{Verbatim}[commandchars=\\\{\}]
{\color{outcolor}Out[{\color{outcolor}33}]:} [('gravity\_moon', 1.635), ('e', 2.3), ('pi', 3.1416)]
\end{Verbatim}
            
    \subsection{Ranges}\label{ranges}

Ranges are "lists" of numbers which can be defined with one instruction.

    \begin{Verbatim}[commandchars=\\\{\}]
{\color{incolor}In [{\color{incolor}34}]:} \PY{n+nb}{range}\PY{p}{(}\PY{l+m+mi}{10}\PY{p}{)}
\end{Verbatim}


\begin{Verbatim}[commandchars=\\\{\}]
{\color{outcolor}Out[{\color{outcolor}34}]:} range(0, 10)
\end{Verbatim}
            
    \begin{Verbatim}[commandchars=\\\{\}]
{\color{incolor}In [{\color{incolor}35}]:} \PY{n}{r} \PY{o}{=} \PY{n+nb}{range}\PY{p}{(}\PY{l+m+mi}{10}\PY{p}{)}
         \PY{n+nb}{type}\PY{p}{(}\PY{n}{r}\PY{p}{)}
\end{Verbatim}


\begin{Verbatim}[commandchars=\\\{\}]
{\color{outcolor}Out[{\color{outcolor}35}]:} range
\end{Verbatim}
            
    Converting a range into a list.

    \begin{Verbatim}[commandchars=\\\{\}]
{\color{incolor}In [{\color{incolor}36}]:} \PY{n}{r} \PY{o}{=} \PY{n+nb}{list}\PY{p}{(}\PY{n}{r}\PY{p}{)}
         \PY{n}{r}
\end{Verbatim}


\begin{Verbatim}[commandchars=\\\{\}]
{\color{outcolor}Out[{\color{outcolor}36}]:} [0, 1, 2, 3, 4, 5, 6, 7, 8, 9]
\end{Verbatim}
            
    Ranges between numbers.

    \begin{Verbatim}[commandchars=\\\{\}]
{\color{incolor}In [{\color{incolor}37}]:} \PY{n+nb}{range}\PY{p}{(}\PY{l+m+mi}{5}\PY{p}{,}\PY{l+m+mi}{10}\PY{p}{)}
\end{Verbatim}


\begin{Verbatim}[commandchars=\\\{\}]
{\color{outcolor}Out[{\color{outcolor}37}]:} range(5, 10)
\end{Verbatim}
            
    \begin{Verbatim}[commandchars=\\\{\}]
{\color{incolor}In [{\color{incolor}38}]:} \PY{c+c1}{\PYZsh{} Setting a range with a step}
         \PY{n+nb}{list}\PY{p}{(}\PY{n+nb}{range}\PY{p}{(}\PY{l+m+mi}{5}\PY{p}{,}\PY{l+m+mi}{10}\PY{p}{,}\PY{l+m+mi}{2}\PY{p}{)}\PY{p}{)}
\end{Verbatim}


\begin{Verbatim}[commandchars=\\\{\}]
{\color{outcolor}Out[{\color{outcolor}38}]:} [5, 7, 9]
\end{Verbatim}
            
    \begin{Verbatim}[commandchars=\\\{\}]
{\color{incolor}In [{\color{incolor}39}]:} \PY{n+nb}{list}\PY{p}{(}\PY{n+nb}{range}\PY{p}{(}\PY{l+m+mi}{9}\PY{p}{,}\PY{l+m+mi}{1}\PY{p}{)}\PY{p}{)}
\end{Verbatim}


\begin{Verbatim}[commandchars=\\\{\}]
{\color{outcolor}Out[{\color{outcolor}39}]:} []
\end{Verbatim}
            
    \begin{Verbatim}[commandchars=\\\{\}]
{\color{incolor}In [{\color{incolor}40}]:} \PY{n+nb}{list}\PY{p}{(}\PY{n+nb}{range}\PY{p}{(}\PY{l+m+mi}{9}\PY{p}{,}\PY{l+m+mi}{1}\PY{p}{,}\PY{o}{\PYZhy{}}\PY{l+m+mi}{1}\PY{p}{)}\PY{p}{)}
\end{Verbatim}


\begin{Verbatim}[commandchars=\\\{\}]
{\color{outcolor}Out[{\color{outcolor}40}]:} [9, 8, 7, 6, 5, 4, 3, 2]
\end{Verbatim}
            
    \subsection{Sets}\label{sets}

Unordered collection of elements.

    \begin{Verbatim}[commandchars=\\\{\}]
{\color{incolor}In [{\color{incolor}41}]:} \PY{n}{basket} \PY{o}{=} \PY{p}{\PYZob{}}\PY{l+s+s1}{\PYZsq{}}\PY{l+s+s1}{apple}\PY{l+s+s1}{\PYZsq{}}\PY{p}{,} \PY{l+s+s1}{\PYZsq{}}\PY{l+s+s1}{orange}\PY{l+s+s1}{\PYZsq{}}\PY{p}{,} \PY{l+s+s1}{\PYZsq{}}\PY{l+s+s1}{apple}\PY{l+s+s1}{\PYZsq{}}\PY{p}{,} \PY{l+s+s1}{\PYZsq{}}\PY{l+s+s1}{pear}\PY{l+s+s1}{\PYZsq{}}\PY{p}{,} \PY{l+s+s1}{\PYZsq{}}\PY{l+s+s1}{orange}\PY{l+s+s1}{\PYZsq{}}\PY{p}{,} \PY{l+s+s1}{\PYZsq{}}\PY{l+s+s1}{banana}\PY{l+s+s1}{\PYZsq{}}\PY{p}{\PYZcb{}}
         \PY{n+nb}{print}\PY{p}{(}\PY{n}{basket}\PY{p}{)}
\end{Verbatim}


    \begin{Verbatim}[commandchars=\\\{\}]
\{'orange', 'apple', 'pear', 'banana'\}

    \end{Verbatim}

    \begin{Verbatim}[commandchars=\\\{\}]
{\color{incolor}In [{\color{incolor}42}]:} \PY{c+c1}{\PYZsh{} check if an element is in a set}
         \PY{l+s+s1}{\PYZsq{}}\PY{l+s+s1}{grape}\PY{l+s+s1}{\PYZsq{}} \PY{o+ow}{in} \PY{n}{basket}
\end{Verbatim}


\begin{Verbatim}[commandchars=\\\{\}]
{\color{outcolor}Out[{\color{outcolor}42}]:} False
\end{Verbatim}
            
    \subsection{Dictionaries}\label{dictionaries}

"Lists" indexed with a key.

    \begin{Verbatim}[commandchars=\\\{\}]
{\color{incolor}In [{\color{incolor}43}]:} \PY{n}{dictStudent} \PY{o}{=} \PY{p}{\PYZob{}}\PY{l+s+s1}{\PYZsq{}}\PY{l+s+s1}{Name}\PY{l+s+s1}{\PYZsq{}}\PY{p}{:} \PY{l+s+s1}{\PYZsq{}}\PY{l+s+s1}{Alex}\PY{l+s+s1}{\PYZsq{}}\PY{p}{,} \PY{l+s+s1}{\PYZsq{}}\PY{l+s+s1}{Age}\PY{l+s+s1}{\PYZsq{}}\PY{p}{:} \PY{l+m+mi}{27}\PY{p}{,} \PY{l+s+s1}{\PYZsq{}}\PY{l+s+s1}{Grade}\PY{l+s+s1}{\PYZsq{}}\PY{p}{:} \PY{l+s+s1}{\PYZsq{}}\PY{l+s+s1}{A}\PY{l+s+s1}{\PYZsq{}}\PY{p}{\PYZcb{}}
         \PY{n}{dictStudent}
\end{Verbatim}


\begin{Verbatim}[commandchars=\\\{\}]
{\color{outcolor}Out[{\color{outcolor}43}]:} \{'Age': 27, 'Grade': 'A', 'Name': 'Alex'\}
\end{Verbatim}
            
    \begin{Verbatim}[commandchars=\\\{\}]
{\color{incolor}In [{\color{incolor}44}]:} \PY{n+nb}{len}\PY{p}{(}\PY{n}{dictStudent}\PY{p}{)}
\end{Verbatim}


\begin{Verbatim}[commandchars=\\\{\}]
{\color{outcolor}Out[{\color{outcolor}44}]:} 3
\end{Verbatim}
            
    \begin{Verbatim}[commandchars=\\\{\}]
{\color{incolor}In [{\color{incolor}45}]:} \PY{c+c1}{\PYZsh{} Requesting a specific element from the dictionary by using the key}
         \PY{n+nb}{print}\PY{p}{(}\PY{n}{dictStudent}\PY{p}{[}\PY{l+s+s1}{\PYZsq{}}\PY{l+s+s1}{Name}\PY{l+s+s1}{\PYZsq{}}\PY{p}{]}\PY{p}{)}
\end{Verbatim}


    \begin{Verbatim}[commandchars=\\\{\}]
Alex

    \end{Verbatim}

    \subsection{Numpy Arrays}\label{numpy-arrays}

Numpy is a python module which allows the use of "numpy arrays", which
are easy to use vectors and matrices of numbers.

    \begin{Verbatim}[commandchars=\\\{\}]
{\color{incolor}In [{\color{incolor}46}]:} \PY{o}{!}pip install numpy
\end{Verbatim}


    \begin{Verbatim}[commandchars=\\\{\}]
Requirement already satisfied: numpy in d:\textbackslash{}anaconda3\textbackslash{}lib\textbackslash{}site-packages

    \end{Verbatim}

    \begin{Verbatim}[commandchars=\\\{\}]
You are using pip version 9.0.1, however version 18.1 is available.
You should consider upgrading via the 'python -m pip install --upgrade pip' command.

    \end{Verbatim}

    \begin{Verbatim}[commandchars=\\\{\}]
{\color{incolor}In [{\color{incolor}47}]:} \PY{k+kn}{import} \PY{n+nn}{numpy} \PY{k}{as} \PY{n+nn}{np}
         
         \PY{c+c1}{\PYZsh{} Defining a vector}
         \PY{n}{N} \PY{o}{=} \PY{n}{np}\PY{o}{.}\PY{n}{array}\PY{p}{(}\PY{p}{[}\PY{l+m+mi}{1}\PY{p}{,}\PY{l+m+mi}{3}\PY{p}{,}\PY{l+m+mi}{5}\PY{p}{,}\PY{l+m+mi}{7}\PY{p}{,}\PY{l+m+mi}{9}\PY{p}{]}\PY{p}{)}
         \PY{n+nb}{print}\PY{p}{(}\PY{n}{N}\PY{p}{)}
\end{Verbatim}


    \begin{Verbatim}[commandchars=\\\{\}]
[1 3 5 7 9]

    \end{Verbatim}

    Operations for numpy arrays.

    \begin{Verbatim}[commandchars=\\\{\}]
{\color{incolor}In [{\color{incolor}48}]:} \PY{n}{N}\PY{o}{+}\PY{l+m+mi}{1}
\end{Verbatim}


\begin{Verbatim}[commandchars=\\\{\}]
{\color{outcolor}Out[{\color{outcolor}48}]:} array([ 2,  4,  6,  8, 10])
\end{Verbatim}
            
    \begin{Verbatim}[commandchars=\\\{\}]
{\color{incolor}In [{\color{incolor}49}]:} \PY{n+nb}{sum}\PY{p}{(}\PY{n}{N}\PY{p}{)}
\end{Verbatim}


\begin{Verbatim}[commandchars=\\\{\}]
{\color{outcolor}Out[{\color{outcolor}49}]:} 25
\end{Verbatim}
            
    \begin{Verbatim}[commandchars=\\\{\}]
{\color{incolor}In [{\color{incolor}50}]:} \PY{n}{N}\PY{o}{.}\PY{n}{shape}
\end{Verbatim}


\begin{Verbatim}[commandchars=\\\{\}]
{\color{outcolor}Out[{\color{outcolor}50}]:} (5,)
\end{Verbatim}
            
    Different to a list comparison, a comparison between numpy arrays
delivers a one to one output.

    \begin{Verbatim}[commandchars=\\\{\}]
{\color{incolor}In [{\color{incolor}51}]:} \PY{n}{np}\PY{o}{.}\PY{n}{array}\PY{p}{(}\PY{p}{[}\PY{l+m+mi}{1}\PY{p}{,}\PY{l+m+mi}{2}\PY{p}{,}\PY{l+m+mi}{3}\PY{p}{]}\PY{p}{)}\PY{o}{==}\PY{n}{np}\PY{o}{.}\PY{n}{array}\PY{p}{(}\PY{p}{[}\PY{l+m+mi}{1}\PY{p}{,}\PY{l+m+mi}{3}\PY{p}{,}\PY{l+m+mi}{3}\PY{p}{]}\PY{p}{)}
\end{Verbatim}


\begin{Verbatim}[commandchars=\\\{\}]
{\color{outcolor}Out[{\color{outcolor}51}]:} array([ True, False,  True])
\end{Verbatim}
            
    Using comparisons in numpy arrays.

    \begin{Verbatim}[commandchars=\\\{\}]
{\color{incolor}In [{\color{incolor}52}]:} \PY{n}{N}\PY{o}{\PYZgt{}}\PY{l+m+mi}{6}
\end{Verbatim}


\begin{Verbatim}[commandchars=\\\{\}]
{\color{outcolor}Out[{\color{outcolor}52}]:} array([False, False, False,  True,  True])
\end{Verbatim}
            
    \begin{Verbatim}[commandchars=\\\{\}]
{\color{incolor}In [{\color{incolor}53}]:} \PY{n}{N}\PY{p}{[}\PY{n}{N}\PY{o}{\PYZgt{}}\PY{l+m+mi}{6}\PY{p}{]}
\end{Verbatim}


\begin{Verbatim}[commandchars=\\\{\}]
{\color{outcolor}Out[{\color{outcolor}53}]:} array([7, 9])
\end{Verbatim}
            
    Defining "empty" vectors.

    \begin{Verbatim}[commandchars=\\\{\}]
{\color{incolor}In [{\color{incolor}54}]:} \PY{n}{zero\PYZus{}vector} \PY{o}{=} \PY{n}{np}\PY{o}{.}\PY{n}{zeros}\PY{p}{(}\PY{l+m+mi}{5}\PY{p}{)}
         \PY{n}{zero\PYZus{}vector}
\end{Verbatim}


\begin{Verbatim}[commandchars=\\\{\}]
{\color{outcolor}Out[{\color{outcolor}54}]:} array([0., 0., 0., 0., 0.])
\end{Verbatim}
            
    Defining matrices.

    \begin{Verbatim}[commandchars=\\\{\}]
{\color{incolor}In [{\color{incolor}55}]:} \PY{n}{matrix1} \PY{o}{=} \PY{n}{np}\PY{o}{.}\PY{n}{array}\PY{p}{(}\PY{p}{[}\PY{p}{[}\PY{l+m+mi}{1}\PY{p}{,}\PY{l+m+mi}{2}\PY{p}{]}\PY{p}{,}\PY{p}{[}\PY{l+m+mi}{3}\PY{p}{,}\PY{l+m+mi}{4}\PY{p}{]}\PY{p}{]}\PY{p}{)}
         \PY{n}{matrix1}
\end{Verbatim}


\begin{Verbatim}[commandchars=\\\{\}]
{\color{outcolor}Out[{\color{outcolor}55}]:} array([[1, 2],
                [3, 4]])
\end{Verbatim}
            
    Defining "empty" matrices.

    \begin{Verbatim}[commandchars=\\\{\}]
{\color{incolor}In [{\color{incolor}56}]:} \PY{n}{zero\PYZus{}matrix} \PY{o}{=} \PY{n}{np}\PY{o}{.}\PY{n}{zeros}\PY{p}{(}\PY{p}{(}\PY{l+m+mi}{5}\PY{p}{,}\PY{l+m+mi}{3}\PY{p}{)}\PY{p}{)}
         \PY{n+nb}{print}\PY{p}{(}\PY{n}{zero\PYZus{}matrix}\PY{p}{,} \PY{n}{zero\PYZus{}matrix}\PY{o}{.}\PY{n}{shape}\PY{p}{)} 
\end{Verbatim}


    \begin{Verbatim}[commandchars=\\\{\}]
[[0. 0. 0.]
 [0. 0. 0.]
 [0. 0. 0.]
 [0. 0. 0.]
 [0. 0. 0.]] (5, 3)

    \end{Verbatim}

    Calculating the transpose matrix.

    \begin{Verbatim}[commandchars=\\\{\}]
{\color{incolor}In [{\color{incolor}57}]:} \PY{n}{zero\PYZus{}matrix\PYZus{}transpose} \PY{o}{=} \PY{n}{zero\PYZus{}matrix}\PY{o}{.}\PY{n}{transpose}\PY{p}{(}\PY{p}{)}
         \PY{n+nb}{print}\PY{p}{(}\PY{n}{zero\PYZus{}matrix\PYZus{}transpose}\PY{p}{,} \PY{n}{zero\PYZus{}matrix\PYZus{}transpose}\PY{o}{.}\PY{n}{shape}\PY{p}{)} 
\end{Verbatim}


    \begin{Verbatim}[commandchars=\\\{\}]
[[0. 0. 0. 0. 0.]
 [0. 0. 0. 0. 0.]
 [0. 0. 0. 0. 0.]] (3, 5)

    \end{Verbatim}

    \subsection{Loops \& Conditions}\label{loops-conditions}

This section contains a series of examples of how for and if statements
can be used to generate code in Python.

    \begin{Verbatim}[commandchars=\\\{\}]
{\color{incolor}In [{\color{incolor}58}]:} \PY{c+c1}{\PYZsh{} print numbers in a list}
         \PY{k}{for} \PY{n}{i} \PY{o+ow}{in} \PY{n+nb}{range}\PY{p}{(}\PY{l+m+mi}{10}\PY{p}{)}\PY{p}{:}
                 \PY{n+nb}{print}\PY{p}{(}\PY{l+s+s1}{\PYZsq{}}\PY{l+s+s1}{The number is: }\PY{l+s+s1}{\PYZsq{}}\PY{p}{,}\PY{n}{i}\PY{p}{)}
\end{Verbatim}


    \begin{Verbatim}[commandchars=\\\{\}]
The number is:  0
The number is:  1
The number is:  2
The number is:  3
The number is:  4
The number is:  5
The number is:  6
The number is:  7
The number is:  8
The number is:  9

    \end{Verbatim}

    \begin{Verbatim}[commandchars=\\\{\}]
{\color{incolor}In [{\color{incolor}59}]:} \PY{c+c1}{\PYZsh{} print even numbers in a list}
         \PY{k}{for} \PY{n}{i} \PY{o+ow}{in} \PY{n+nb}{range}\PY{p}{(}\PY{l+m+mi}{10}\PY{p}{)}\PY{p}{:}
             \PY{k}{if} \PY{n}{i}\PY{o}{\PYZpc{}}\PY{k}{2} ==0:
                 \PY{n+nb}{print}\PY{p}{(}\PY{l+s+s1}{\PYZsq{}}\PY{l+s+s1}{The even numer: }\PY{l+s+s1}{\PYZsq{}}\PY{p}{,}\PY{n}{i}\PY{p}{)}
\end{Verbatim}


    \begin{Verbatim}[commandchars=\\\{\}]
The even numer:  0
The even numer:  2
The even numer:  4
The even numer:  6
The even numer:  8

    \end{Verbatim}

    \begin{Verbatim}[commandchars=\\\{\}]
{\color{incolor}In [{\color{incolor}60}]:} \PY{c+c1}{\PYZsh{} loop over a list of words and calculate their length}
         \PY{n}{words} \PY{o}{=} \PY{p}{[}\PY{l+s+s1}{\PYZsq{}}\PY{l+s+s1}{Robert}\PY{l+s+s1}{\PYZsq{}}\PY{p}{,}\PY{l+s+s1}{\PYZsq{}}\PY{l+s+s1}{Gordon}\PY{l+s+s1}{\PYZsq{}}\PY{p}{,}\PY{l+s+s1}{\PYZsq{}}\PY{l+s+s1}{University}\PY{l+s+s1}{\PYZsq{}}\PY{p}{,}\PY{l+s+s1}{\PYZsq{}}\PY{l+s+s1}{Data}\PY{l+s+s1}{\PYZsq{}}\PY{p}{,}\PY{l+s+s1}{\PYZsq{}}\PY{l+s+s1}{Science}\PY{l+s+s1}{\PYZsq{}}\PY{p}{]}
         \PY{k}{for} \PY{n}{w} \PY{o+ow}{in} \PY{n}{words}\PY{p}{:}
             \PY{n+nb}{print}\PY{p}{(}\PY{n}{w}\PY{p}{,}\PY{n+nb}{len}\PY{p}{(}\PY{n}{w}\PY{p}{)}\PY{p}{)}
\end{Verbatim}


    \begin{Verbatim}[commandchars=\\\{\}]
Robert 6
Gordon 6
University 10
Data 4
Science 7

    \end{Verbatim}

    \begin{Verbatim}[commandchars=\\\{\}]
{\color{incolor}In [{\color{incolor}61}]:} \PY{n}{words} \PY{o}{=} \PY{p}{[}\PY{l+s+s1}{\PYZsq{}}\PY{l+s+s1}{Robert}\PY{l+s+s1}{\PYZsq{}}\PY{p}{,}\PY{l+s+s1}{\PYZsq{}}\PY{l+s+s1}{Gordon}\PY{l+s+s1}{\PYZsq{}}\PY{p}{,}\PY{l+s+s1}{\PYZsq{}}\PY{l+s+s1}{University}\PY{l+s+s1}{\PYZsq{}}\PY{p}{,}\PY{l+s+s1}{\PYZsq{}}\PY{l+s+s1}{Data}\PY{l+s+s1}{\PYZsq{}}\PY{p}{,}\PY{l+s+s1}{\PYZsq{}}\PY{l+s+s1}{Science}\PY{l+s+s1}{\PYZsq{}}\PY{p}{]}
         \PY{k}{for} \PY{n}{i} \PY{o+ow}{in} \PY{n+nb}{range}\PY{p}{(}\PY{n+nb}{len}\PY{p}{(}\PY{n}{words}\PY{p}{)}\PY{p}{)}\PY{p}{:}
             \PY{n+nb}{print}\PY{p}{(}\PY{n}{i}\PY{p}{,} \PY{n}{words}\PY{p}{[}\PY{n}{i}\PY{p}{]}\PY{p}{)}
\end{Verbatim}


    \begin{Verbatim}[commandchars=\\\{\}]
0 Robert
1 Gordon
2 University
3 Data
4 Science

    \end{Verbatim}

    \begin{Verbatim}[commandchars=\\\{\}]
{\color{incolor}In [{\color{incolor}62}]:} \PY{k}{for} \PY{n}{i} \PY{o+ow}{in} \PY{n+nb}{range}\PY{p}{(}\PY{l+m+mi}{5}\PY{p}{)}\PY{p}{:}
             \PY{n+nb}{print} \PY{p}{(}\PY{l+s+s1}{\PYZsq{}}\PY{l+s+s1}{square of }\PY{l+s+s1}{\PYZsq{}}\PY{o}{+}\PY{n+nb}{str}\PY{p}{(}\PY{n}{i}\PY{p}{)}\PY{o}{+}\PY{l+s+s1}{\PYZsq{}}\PY{l+s+s1}{ is}\PY{l+s+s1}{\PYZsq{}}\PY{p}{,}\PY{n}{i}\PY{o}{*}\PY{o}{*}\PY{l+m+mi}{2}\PY{p}{)}
\end{Verbatim}


    \begin{Verbatim}[commandchars=\\\{\}]
square of 0 is 0
square of 1 is 1
square of 2 is 4
square of 3 is 9
square of 4 is 16

    \end{Verbatim}

    \begin{Verbatim}[commandchars=\\\{\}]
{\color{incolor}In [{\color{incolor}63}]:} \PY{c+c1}{\PYZsh{} Loop over a dictionary}
         \PY{k}{for} \PY{n}{k}\PY{p}{,} \PY{n}{v} \PY{o+ow}{in} \PY{n}{dictStudent}\PY{o}{.}\PY{n}{items}\PY{p}{(}\PY{p}{)}\PY{p}{:}
              \PY{n+nb}{print}\PY{p}{(}\PY{n}{k}\PY{p}{,} \PY{n}{v}\PY{p}{)}
\end{Verbatim}


    \begin{Verbatim}[commandchars=\\\{\}]
Name Alex
Age 27
Grade A

    \end{Verbatim}

    \begin{Verbatim}[commandchars=\\\{\}]
{\color{incolor}In [{\color{incolor}64}]:} \PY{c+c1}{\PYZsh{} Detect numbers}
         \PY{n}{x} \PY{o}{=} \PY{n+nb}{int}\PY{p}{(}\PY{n+nb}{input}\PY{p}{(}\PY{l+s+s2}{\PYZdq{}}\PY{l+s+s2}{Please enter number: }\PY{l+s+s2}{\PYZdq{}}\PY{p}{)}\PY{p}{)}
         \PY{k}{if} \PY{n}{x} \PY{o}{\PYZlt{}} \PY{l+m+mi}{0}\PY{p}{:}
                 \PY{n+nb}{print}\PY{p}{(}\PY{l+s+s1}{\PYZsq{}}\PY{l+s+s1}{Negative}\PY{l+s+s1}{\PYZsq{}}\PY{p}{)}
         \PY{k}{elif} \PY{n}{x} \PY{o}{==} \PY{l+m+mi}{0}\PY{p}{:}
                 \PY{n+nb}{print}\PY{p}{(}\PY{l+s+s1}{\PYZsq{}}\PY{l+s+s1}{Zero}\PY{l+s+s1}{\PYZsq{}}\PY{p}{)}
         \PY{k}{elif} \PY{n}{x} \PY{o}{==} \PY{l+m+mi}{1}\PY{p}{:}
                 \PY{n+nb}{print}\PY{p}{(}\PY{l+s+s1}{\PYZsq{}}\PY{l+s+s1}{Unit}\PY{l+s+s1}{\PYZsq{}}\PY{p}{)}
         \PY{k}{else}\PY{p}{:}
                 \PY{n+nb}{print}\PY{p}{(}\PY{l+s+s1}{\PYZsq{}}\PY{l+s+s1}{Larger than 1}\PY{l+s+s1}{\PYZsq{}}\PY{p}{)}
\end{Verbatim}


    \begin{Verbatim}[commandchars=\\\{\}]
Please enter number: 9
Larger than 1

    \end{Verbatim}

    \subsection{Functions}\label{functions}

To simplify code or when a series of instructions will be used several
times, it is preferable to store them in a function.

    \begin{Verbatim}[commandchars=\\\{\}]
{\color{incolor}In [{\color{incolor}65}]:} \PY{c+c1}{\PYZsh{} Defining a function to calculate the distance between two points}
         \PY{k+kn}{import} \PY{n+nn}{numpy} \PY{k}{as} \PY{n+nn}{np}
         
         \PY{k}{def} \PY{n+nf}{distPoints}\PY{p}{(}\PY{n}{p1}\PY{p}{,}\PY{n}{p2}\PY{p}{)}\PY{p}{:}
             \PY{l+s+sd}{\PYZsq{}\PYZsq{}\PYZsq{}This function calculates the distance between two points p1 and p2 represented as 2\PYZhy{}dim tuples.\PYZsq{}\PYZsq{}\PYZsq{}}
             \PY{k}{return} \PY{n}{np}\PY{o}{.}\PY{n}{sqrt}\PY{p}{(}\PY{n}{np}\PY{o}{.}\PY{n}{power}\PY{p}{(}\PY{n}{p1}\PY{p}{[}\PY{l+m+mi}{0}\PY{p}{]}\PY{o}{\PYZhy{}}\PY{n}{p2}\PY{p}{[}\PY{l+m+mi}{0}\PY{p}{]}\PY{p}{,} \PY{l+m+mi}{2}\PY{p}{)}\PY{o}{+}\PY{n}{np}\PY{o}{.}\PY{n}{power}\PY{p}{(}\PY{n}{p1}\PY{p}{[}\PY{l+m+mi}{1}\PY{p}{]}\PY{o}{\PYZhy{}}\PY{n}{p2}\PY{p}{[}\PY{l+m+mi}{1}\PY{p}{]}\PY{p}{,} \PY{l+m+mi}{2}\PY{p}{)}\PY{p}{)}
         
         \PY{c+c1}{\PYZsh{} Use the function}
         \PY{n}{p1} \PY{o}{=} \PY{p}{(}\PY{l+m+mi}{3}\PY{p}{,}\PY{l+m+mi}{4}\PY{p}{)}
         \PY{n}{p2} \PY{o}{=} \PY{p}{(}\PY{l+m+mi}{1}\PY{p}{,}\PY{l+m+mi}{2}\PY{p}{)}
         \PY{n}{dist} \PY{o}{=} \PY{n}{distPoints}\PY{p}{(}\PY{n}{p1}\PY{p}{,}\PY{n}{p2}\PY{p}{)}
         \PY{n+nb}{print}\PY{p}{(}\PY{n}{dist}\PY{p}{)}
\end{Verbatim}


    \begin{Verbatim}[commandchars=\\\{\}]
2.8284271247461903

    \end{Verbatim}

    \begin{Verbatim}[commandchars=\\\{\}]
{\color{incolor}In [{\color{incolor}66}]:} \PY{c+c1}{\PYZsh{} Defining a function to calculate Fibonacci series numbers}
         
         \PY{k}{def} \PY{n+nf}{fib}\PY{p}{(}\PY{n}{n}\PY{p}{)}\PY{p}{:}
              \PY{l+s+sd}{\PYZdq{}\PYZdq{}\PYZdq{}Return a list containing the Fibonacci series up to n.\PYZdq{}\PYZdq{}\PYZdq{}}
              \PY{n}{result} \PY{o}{=} \PY{p}{[}\PY{p}{]}
              \PY{n}{a} \PY{o}{=} \PY{l+m+mi}{0}
              \PY{n}{b} \PY{o}{=} \PY{l+m+mi}{1}
              \PY{k}{while} \PY{n}{a} \PY{o}{\PYZlt{}} \PY{n}{n}\PY{p}{:}
                  \PY{n}{result}\PY{o}{.}\PY{n}{append}\PY{p}{(}\PY{n}{a}\PY{p}{)}
                  \PY{n}{a}\PY{p}{,} \PY{n}{b} \PY{o}{=} \PY{n}{b}\PY{p}{,} \PY{n}{a}\PY{o}{+}\PY{n}{b}
              \PY{k}{return} \PY{n}{result}
         
         \PY{c+c1}{\PYZsh{} Call the function}
         \PY{n}{fib}\PY{p}{(}\PY{l+m+mi}{10}\PY{p}{)}
\end{Verbatim}


\begin{Verbatim}[commandchars=\\\{\}]
{\color{outcolor}Out[{\color{outcolor}66}]:} [0, 1, 1, 2, 3, 5, 8]
\end{Verbatim}
            
    \begin{Verbatim}[commandchars=\\\{\}]
{\color{incolor}In [{\color{incolor}67}]:} \PY{c+c1}{\PYZsh{} Why are these versions not working?}
         \PY{k}{def} \PY{n+nf}{fib2}\PY{p}{(}\PY{n}{n}\PY{p}{)}\PY{p}{:}
              \PY{l+s+sd}{\PYZdq{}\PYZdq{}\PYZdq{}Return a list containing the Fibonacci series up to n.\PYZdq{}\PYZdq{}\PYZdq{}}
              \PY{n}{result} \PY{o}{=} \PY{p}{[}\PY{p}{]}
              \PY{n}{a} \PY{o}{=} \PY{l+m+mi}{0}
              \PY{n}{b} \PY{o}{=} \PY{l+m+mi}{1}
              \PY{k}{while} \PY{n}{a} \PY{o}{\PYZlt{}} \PY{n}{n}\PY{p}{:}
                  \PY{n}{result}\PY{o}{.}\PY{n}{append}\PY{p}{(}\PY{n}{a}\PY{p}{)}
                  \PY{n}{a} \PY{o}{=} \PY{n}{b}
                  \PY{n}{b} \PY{o}{=} \PY{n}{a}\PY{o}{+}\PY{n}{b}
              \PY{k}{return} \PY{n}{result}
         \PY{k}{def} \PY{n+nf}{fib3}\PY{p}{(}\PY{n}{n}\PY{p}{)}\PY{p}{:}
              \PY{l+s+sd}{\PYZdq{}\PYZdq{}\PYZdq{}Return a list containing the Fibonacci series up to n.\PYZdq{}\PYZdq{}\PYZdq{}}
              \PY{n}{result} \PY{o}{=} \PY{p}{[}\PY{p}{]}
              \PY{n}{a} \PY{o}{=} \PY{l+m+mi}{0}
              \PY{n}{b} \PY{o}{=} \PY{l+m+mi}{1}
              \PY{k}{while} \PY{n}{a} \PY{o}{\PYZlt{}} \PY{n}{n}\PY{p}{:}
                  \PY{n}{result}\PY{o}{.}\PY{n}{append}\PY{p}{(}\PY{n}{a}\PY{p}{)}
                  \PY{n}{b} \PY{o}{=} \PY{n}{a}\PY{o}{+}\PY{n}{b}
                  \PY{n}{a} \PY{o}{=} \PY{n}{b}
              \PY{k}{return} \PY{n}{result}
         
         \PY{n+nb}{print}\PY{p}{(}\PY{n}{fib2}\PY{p}{(}\PY{l+m+mi}{10}\PY{p}{)}\PY{p}{)}
         \PY{n+nb}{print}\PY{p}{(}\PY{n}{fib3}\PY{p}{(}\PY{l+m+mi}{10}\PY{p}{)}\PY{p}{)}
\end{Verbatim}


    \begin{Verbatim}[commandchars=\\\{\}]
[0, 1, 2, 4, 8]
[0, 1, 2, 4, 8]

    \end{Verbatim}

    \subsection{List Comprehension}\label{list-comprehension}

Python allows to perform a loop and/or a conditional operation in one
line.

    \begin{Verbatim}[commandchars=\\\{\}]
{\color{incolor}In [{\color{incolor}68}]:} \PY{c+c1}{\PYZsh{} Create a list containing the square of numbers from 0 to 9}
         
         \PY{n+nb}{list}\PY{p}{(}\PY{n+nb}{map}\PY{p}{(}\PY{k}{lambda} \PY{n}{x}\PY{p}{:} \PY{n}{x}\PY{o}{*}\PY{o}{*}\PY{l+m+mi}{2}\PY{p}{,} \PY{n+nb}{range}\PY{p}{(}\PY{l+m+mi}{10}\PY{p}{)}\PY{p}{)}\PY{p}{)}
\end{Verbatim}


\begin{Verbatim}[commandchars=\\\{\}]
{\color{outcolor}Out[{\color{outcolor}68}]:} [0, 1, 4, 9, 16, 25, 36, 49, 64, 81]
\end{Verbatim}
            
    \begin{Verbatim}[commandchars=\\\{\}]
{\color{incolor}In [{\color{incolor}69}]:} \PY{c+c1}{\PYZsh{} sum odd numbers from 0 to 9}
         \PY{n+nb}{sum}\PY{p}{(}\PY{p}{[}\PY{n}{i} \PY{k}{for} \PY{n}{i} \PY{o+ow}{in} \PY{n+nb}{range}\PY{p}{(}\PY{l+m+mi}{10}\PY{p}{)} \PY{k}{if} \PY{n}{i}\PY{o}{\PYZpc{}}\PY{k}{2}==1])
\end{Verbatim}


\begin{Verbatim}[commandchars=\\\{\}]
{\color{outcolor}Out[{\color{outcolor}69}]:} 25
\end{Verbatim}
            
    \begin{Verbatim}[commandchars=\\\{\}]
{\color{incolor}In [{\color{incolor}70}]:} \PY{c+c1}{\PYZsh{} Create a list containing tuple pairs of numbers in the lists [1,2,3] and [3,1,4] }
         \PY{c+c1}{\PYZsh{}\PYZsh{} excluding àirs of the same number.}
         \PY{p}{[}\PY{p}{(}\PY{n}{x}\PY{p}{,} \PY{n}{y}\PY{p}{)} \PY{k}{for} \PY{n}{x} \PY{o+ow}{in} \PY{p}{[}\PY{l+m+mi}{1}\PY{p}{,}\PY{l+m+mi}{2}\PY{p}{,}\PY{l+m+mi}{3}\PY{p}{]} \PY{k}{for} \PY{n}{y} \PY{o+ow}{in} \PY{p}{[}\PY{l+m+mi}{3}\PY{p}{,}\PY{l+m+mi}{1}\PY{p}{,}\PY{l+m+mi}{4}\PY{p}{]} \PY{k}{if} \PY{n}{x} \PY{o}{!=} \PY{n}{y}\PY{p}{]}
\end{Verbatim}


\begin{Verbatim}[commandchars=\\\{\}]
{\color{outcolor}Out[{\color{outcolor}70}]:} [(1, 3), (1, 4), (2, 3), (2, 1), (2, 4), (3, 1), (3, 4)]
\end{Verbatim}
            
    \begin{Verbatim}[commandchars=\\\{\}]
{\color{incolor}In [{\color{incolor}71}]:} \PY{c+c1}{\PYZsh{} Transpose a matrix}
         \PY{n}{matrix} \PY{o}{=} \PY{p}{[}
              \PY{p}{[}\PY{l+m+mi}{1}\PY{p}{,} \PY{l+m+mi}{2}\PY{p}{,} \PY{l+m+mi}{3}\PY{p}{,} \PY{l+m+mi}{4}\PY{p}{]}\PY{p}{,}
              \PY{p}{[}\PY{l+m+mi}{5}\PY{p}{,} \PY{l+m+mi}{6}\PY{p}{,} \PY{l+m+mi}{7}\PY{p}{,} \PY{l+m+mi}{8}\PY{p}{]}\PY{p}{,}
              \PY{p}{[}\PY{l+m+mi}{9}\PY{p}{,} \PY{l+m+mi}{10}\PY{p}{,} \PY{l+m+mi}{11}\PY{p}{,} \PY{l+m+mi}{12}\PY{p}{]}\PY{p}{,}\PY{p}{]}
         \PY{n+nb}{print}\PY{p}{(}\PY{n}{matrix}\PY{p}{)}
         \PY{p}{[}\PY{p}{[}\PY{n}{row}\PY{p}{[}\PY{n}{i}\PY{p}{]} \PY{k}{for} \PY{n}{row} \PY{o+ow}{in} \PY{n}{matrix}\PY{p}{]} \PY{k}{for} \PY{n}{i} \PY{o+ow}{in} \PY{n+nb}{range}\PY{p}{(}\PY{l+m+mi}{4}\PY{p}{)}\PY{p}{]}
\end{Verbatim}


    \begin{Verbatim}[commandchars=\\\{\}]
[[1, 2, 3, 4], [5, 6, 7, 8], [9, 10, 11, 12]]

    \end{Verbatim}

\begin{Verbatim}[commandchars=\\\{\}]
{\color{outcolor}Out[{\color{outcolor}71}]:} [[1, 5, 9], [2, 6, 10], [3, 7, 11], [4, 8, 12]]
\end{Verbatim}
            
    \subsection{Typing}\label{typing}

\begin{itemize}
\tightlist
\item
  Python is a dynamic language, which means that type check is performed
  at runtime.
\item
  Command \(x=3\) first creates object 3, then creates variable x and
  then assigns a reference where variable x takes the value of object 3.
\item
  References \textbf{ALWAYS} link to objects, never to other variables.
\end{itemize}

    \begin{Verbatim}[commandchars=\\\{\}]
{\color{incolor}In [{\color{incolor}72}]:} \PY{c+c1}{\PYZsh{} Example with integers}
         \PY{n}{x} \PY{o}{=} \PY{l+m+mi}{3}
         \PY{n}{y} \PY{o}{=} \PY{n}{x}
         \PY{n+nb}{print}\PY{p}{(}\PY{l+s+s1}{\PYZsq{}}\PY{l+s+s1}{Value of x is}\PY{l+s+s1}{\PYZsq{}}\PY{p}{,} \PY{n}{x}\PY{p}{)}
         \PY{n+nb}{print}\PY{p}{(}\PY{l+s+s1}{\PYZsq{}}\PY{l+s+s1}{Value of y is}\PY{l+s+s1}{\PYZsq{}}\PY{p}{,} \PY{n}{y}\PY{p}{)}
\end{Verbatim}


    \begin{Verbatim}[commandchars=\\\{\}]
Value of x is 3
Value of y is 3

    \end{Verbatim}

    \begin{Verbatim}[commandchars=\\\{\}]
{\color{incolor}In [{\color{incolor}73}]:} \PY{c+c1}{\PYZsh{} Since integers are IMMUTABLE, a new object \PYZsq{}2\PYZsq{} is created. Python removes the reference y\PYZhy{}\PYZgt{}3 and creates y\PYZhy{}\PYZgt{}2}
         \PY{n}{y}\PY{o}{=}\PY{n}{y}\PY{o}{\PYZhy{}}\PY{l+m+mi}{1}
         \PY{n+nb}{print}\PY{p}{(}\PY{l+s+s1}{\PYZsq{}}\PY{l+s+s1}{Value of x is}\PY{l+s+s1}{\PYZsq{}}\PY{p}{,} \PY{n}{x}\PY{p}{)}
         \PY{n+nb}{print}\PY{p}{(}\PY{l+s+s1}{\PYZsq{}}\PY{l+s+s1}{Value of y is}\PY{l+s+s1}{\PYZsq{}}\PY{p}{,} \PY{n}{y}\PY{p}{)}
\end{Verbatim}


    \begin{Verbatim}[commandchars=\\\{\}]
Value of x is 3
Value of y is 2

    \end{Verbatim}

    \begin{Verbatim}[commandchars=\\\{\}]
{\color{incolor}In [{\color{incolor}74}]:} \PY{c+c1}{\PYZsh{} Example with lists}
         \PY{n}{x} \PY{o}{=} \PY{p}{[}\PY{l+m+mi}{1}\PY{p}{,}\PY{l+m+mi}{2}\PY{p}{]}
         \PY{n}{y} \PY{o}{=} \PY{n}{x}
         \PY{n+nb}{print}\PY{p}{(}\PY{l+s+s1}{\PYZsq{}}\PY{l+s+s1}{Value of x is}\PY{l+s+s1}{\PYZsq{}}\PY{p}{,} \PY{n}{x}\PY{p}{)}
         \PY{n+nb}{print}\PY{p}{(}\PY{l+s+s1}{\PYZsq{}}\PY{l+s+s1}{Value of y is}\PY{l+s+s1}{\PYZsq{}}\PY{p}{,} \PY{n}{y}\PY{p}{)}
\end{Verbatim}


    \begin{Verbatim}[commandchars=\\\{\}]
Value of x is [1, 2]
Value of y is [1, 2]

    \end{Verbatim}

    \begin{Verbatim}[commandchars=\\\{\}]
{\color{incolor}In [{\color{incolor}75}]:} \PY{c+c1}{\PYZsh{} Since lists are MUTABLE, the command y[0]=0 changed the list as an object, not x!}
         \PY{n}{y}\PY{p}{[}\PY{l+m+mi}{0}\PY{p}{]}\PY{o}{=}\PY{l+m+mi}{0}
         \PY{n+nb}{print}\PY{p}{(}\PY{l+s+s1}{\PYZsq{}}\PY{l+s+s1}{Value of x is}\PY{l+s+s1}{\PYZsq{}}\PY{p}{,} \PY{n}{x}\PY{p}{)}
         \PY{n+nb}{print}\PY{p}{(}\PY{l+s+s1}{\PYZsq{}}\PY{l+s+s1}{Value of y is}\PY{l+s+s1}{\PYZsq{}}\PY{p}{,} \PY{n}{y}\PY{p}{)}
\end{Verbatim}


    \begin{Verbatim}[commandchars=\\\{\}]
Value of x is [0, 2]
Value of y is [0, 2]

    \end{Verbatim}

    \begin{Verbatim}[commandchars=\\\{\}]
{\color{incolor}In [{\color{incolor}76}]:} \PY{c+c1}{\PYZsh{} Solution: copy}
         \PY{n}{x} \PY{o}{=} \PY{p}{[}\PY{l+m+mi}{1}\PY{p}{,}\PY{l+m+mi}{2}\PY{p}{]}
         \PY{n}{y} \PY{o}{=} \PY{n}{x}\PY{o}{.}\PY{n}{copy}\PY{p}{(}\PY{p}{)}
         \PY{n}{y}\PY{p}{[}\PY{l+m+mi}{0}\PY{p}{]}\PY{o}{=}\PY{l+m+mi}{0}
         \PY{n+nb}{print}\PY{p}{(}\PY{l+s+s1}{\PYZsq{}}\PY{l+s+s1}{Value of x is}\PY{l+s+s1}{\PYZsq{}}\PY{p}{,} \PY{n}{x}\PY{p}{)}
         \PY{n+nb}{print}\PY{p}{(}\PY{l+s+s1}{\PYZsq{}}\PY{l+s+s1}{Value of y is}\PY{l+s+s1}{\PYZsq{}}\PY{p}{,} \PY{n}{y}\PY{p}{)}
\end{Verbatim}


    \begin{Verbatim}[commandchars=\\\{\}]
Value of x is [1, 2]
Value of y is [0, 2]

    \end{Verbatim}

    \section{How to convert the notebook into
slides}\label{how-to-convert-the-notebook-into-slides}

    \subsection{Option 1}\label{option-1}

Install the "rise" plug-in in your Anaconda environment. This will
enable a button in the top of jupyter notebook which will let you do a
slideshow.

For more information about rise, check the following link:

\begin{itemize}
\tightlist
\item
  https://github.com/damianavila/RISE
\end{itemize}

    \subsection{Option 2}\label{option-2}

Open a terminal, go to the directory where the notebook is located and
type:

\begin{itemize}
\tightlist
\item
  jupyter nbconvert \textbf{filename.ipynb} -\/-to slides -\/-post serve
\end{itemize}


    % Add a bibliography block to the postdoc
    
    
    
    \end{document}
